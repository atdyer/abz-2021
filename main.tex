\documentclass[runningheads]{llncs}

\newcommand{\dummyfigsmall}[1]{
  \centering
  \fbox{
    \begin{minipage}[c][0.1\textheight][c]{0.5\textwidth}
      \centering{#1}
    \end{minipage}
  }
}

\newcommand{\dummyfiglarge}[1]{
  \centering
  \fbox{
    \begin{minipage}[c][0.35\textheight][c]{0.5\textwidth}
      \centering{#1}
    \end{minipage}
  }
}

\begin{document}

\title{Sterling: A Web-based Visualizer for Relational Languages}
\author{Tristan Dyer \inst{1} \and John Baugh \inst{2}}
\authorrunning{Dyer and Baugh}

\institute{Brown University \and North Carolina State University}

\authorrunning{Dyer and Baugh}

\maketitle

\begin{abstract}
We introduce Sterling, a web-based visualization tool for Alloy that provides enhanced views of relational models and amends some of the recognized shortcomings of its approach to graph layout.
A scripting view gives users the ability to create custom visualizations using modern JavaScript libraries such as D3 and Cytoscape without sacrificing the immediate visual feedback provided by Alloy. This enables users to more easily develop and understand relational models with characteristics beyond those that are purely topological, where space and position matter, for instance.
We demonstrate effective use of the scripting view by presenting custom visualizations developed to enable verification studies of production scientific software.
While development is driven primarily by the Alloy community, other relational logic and model finding tools can also be accommodated in its model-view architecture, which uses a mediator pattern to keep the approach data agnostic. One such tool, Forge, a relational language and tool built for teaching introductory formal methods as Brown University, employs Sterling as its visualizer.

\end{abstract}

% \begin{abstract}
% We introduce Sterling, a web-based visualization tool for Alloy that provides enhanced versions of existing Alloy visualizations and introduces new capabilities to address shortcomings of the visual abstractions typically used to display instances of relational models.
% While development is driven primarily by the needs of the Alloy community, other relational logic and model finding tools are accommodated by its model-view architecture, which uses a mediator pattern to keep the approach data agnostic.
% We describe our design goals and demonstrate effective use of Sterling by presenting custom visualizations of models with inherently spatial characteristics.
% \end{abstract}

\keywords{Alloy \and Sterling \and Formal Methods \and Visualization}

% \section{Notes}
% \begin{enumerate}
%     \item What is motivating what we're doing? Need to establish credibility.
%     \item How are the design goals motivated? Motivated by verification studies in scientific software development.
%     \item Test-bed for rapid prototyping of GUI interfaces.
%     \item We show examples from scientific and traditional domains of
%     \item Ability to bring in JavaScript libraries easily
%     \item For paper: what language can we use to bolster that React, etc. are used a lot and can give us credibility. Maybe productivity in comparison to Java, e.g.
% \end{enumerate}

\section{Introduction and Background}
\label{introduction}

Model finding tools like Alloy~\cite{jackson2012} enable a lightweight approach to design and reasoning about complex software systems. Such tools provide push-button analysis for both checking assertions within bounded scopes, and for generating instances that satisfy a property of interest. An attractive feature of Alloy is the immediate feedback provided by visualizations, allowing users to inspect instances and counterexamples in order to identify design problems. The ability to communicate visual information \emph{intuitively} therefore plays a key role in determining the effectiveness of interactions with the user~\cite{gammaitoni2014}.

The Alloy Analyzer includes a visualizer that can display an instance as a directed graph in which nodes represent atoms and edges represent tuples of relations.
To help the user better understand the instance, basic properties of the graph such as color and shape can be customized, and the graph nodes can manually repositioned to achieve a more clear layout.
Additionally, the graph view supports ``projection'', a feature most commonly used with models of dynamic systems. 
When an instance of such a model is projected over time, the user is able to interactively step through snapshots of individual states in sequence.

Despite these features, common issues, such as the lack of customization of the layout algorithm and inability to drag nodes out of the rows into which they are initially laid out~\cite{couto2018,macedo2019}, make it difficult to interpret instances as they grow in size and complexity.
Furthermore, the graph layout is recalculated any time a new instance is generated or the projection is changed, and so the user is forced to reinterpret the entire graph if, for example, they are stepping through the state atoms in sequence~\cite{couto2018,misue1995,zaman2013}.
While various approaches have been proposed to address these and other issues, either by extending the existing visualizer~\cite{zaman2013} or introducing new tools~\cite{couto2018,macedo2019,gammaitoni2014}, our own experience using Alloy in the field of scientific computing has highlighted the need for an approach that facilitates the clear expression of \emph{spatial} relationships---not just topological ones---and to \emph{consistency} in those relationships when dynamic updates occur, as they do in problems with time-varying state.

% \section{Introduction}
% \label{introduction}

% Model finding tools like Alloy~\cite{jackson2012} enable a lightweight approach to design and reasoning about complex software systems. Such tools provide push-button analysis for both checking assertions within bounded scopes, and for generating instances that satisfy a property of interest. An attractive feature of Alloy is the immediate feedback provided by visualizations, allowing users to inspect instances and counterexamples in order to identify design problems. The ability to communicate visual information \emph{intuitively} therefore plays a key role in determining the effectiveness of interactions with the user~\cite{gammaitoni2014}.

% \section{Visualization in Alloy}
% \label{alloy-vis}

% The Alloy Analyzer includes a visualizer that can display an instance in one of four possible views: text, table, tree, and, perhaps the most frequently used, graph. 
% In that view, the instance is rendered as a directed graph in which nodes represent atoms and edges represent tuples of relations. 
% Basic properties of the graph, such as color and shape, can be customized manually or by using themes, which, in combination with the ability to manually reposition nodes, allows the user to interactively explore the graph so that they can better understand the instance. 
% Additionally, the graph view supports ``projection'', a feature most commonly used with models of dynamic systems. 
% When an instance of such a model is projected over time, the user is able to interactively step through snapshots of individual states in sequence.

% Despite these features, certain limitations of the visualizer make it difficult to use as instances increase in size and complexity. 
% Common issues identified in the literature include the layout algorithm, which provides no option for customization by the user and rigidly arranges graph nodes into rows~\cite{couto2018,macedo2019}, and the low-level nature of the graph representation, which can become difficult to interpret as models become more complex~\cite{gammaitoni2014}. 
% Furthermore, the graph layout is recalculated any time a new instance is generated or the projection is changed, and so the user must reinterpret the entire graph if, for example, they are stepping through the state atoms in sequence~\cite{couto2018,misue1995,zaman2013}.

% Various approaches have been proposed to address these and other issues, either by extending Alloy's visualizer or by introducing new tools. 
% Those extending the existing visualizer may, for instance, improve the visualization by automatically determining projections and attributes using contextual clues from the instance~\cite{zaman2013}.
% Those introducing new tools may, e.g., provide a wider range of layout algorithms~\cite{couto2018,macedo2019}, or extend the Alloy language so the user can provide visualization directives as part of the model~\cite{gammaitoni2014}.

%Our own experience using Alloy in the field of scientific computing, however, has highlighted the need for an approach that facilitates the clear expression of \emph{spatial} relationships---not just topological ones---and to \emph{consistency} in those relationships when dynamic updates occur, as they do in problems with time-varying state.
In one prior study, the authors use Alloy to explore implementation choices and ensure soundness of an extension made to an ocean circulation model called ADCIRC~\cite{baugh-scp-2018}.
In the models, finite element mesh topologies are constrained to include only those that have a planar embedding, ensuring the mesh, which can be thought of as a triangulation of a surface, is physically meaningful. 
Inferring the planar embedding given only the topologies of the model relations proved difficult for the authors, stating that ``more than any extension to Alloy, what would have benefited our study most is a tool capable of automatically producing planar embeddings of meshes from Alloy instances, which proved to be tedious to do by hand.''
In another study, we use Alloy to verify the full functional correctness within a bounded scope of sparse matrix formats~\cite{dyer2019}, which use array indirection and other machinery to avoid storing zeroes.
Dense matrices are modeled as relations mapping indices to values, and they can contain dozens of tuples, resulting in a visualization cluttered with edges.
As we move to sparse matrices and the dynamic state changes that accompany operations like matrix multiplication, the visualizations become nearly impossible to interpret.


\section{Sterling Design Goals and Architecture}
\label{sterling}

% Based on both the strengths and shortcomings of the existing visualizations, and drawing inspiration from the Thirteen Principles of Display Design~\cite{wickens2003} (as indicated by \emph{italics}), our approach to instance visualization is characterized by the following design goals: (1) based on the \emph{principle of consistency} Sterling should provide a user interface similar to that of the Alloy visualizer when possible; (2) Sterling should \emph{minimize information access cost} by providing immediate visual feedback and minimizing the effort required to create legible visualizations; and (3) based on the \emph{principle of pictorial realism} ...and common requirements needed in our research in scientific computing...Sterling should provide functionality for creating domain specific visualizations....that express spatial relationships

Motivated by these studies, we developed an approach to instance visualization that draws on the strengths of existing visualizations and draws inspiration from the Thirteen Principles of Display Design~\cite{wickens2003}, as characterized by the following design goals: (1) based on the \emph{principle of consistency} a visualizer should provide a user interface similar to that of the Alloy visualizer when possible; (2) a visualizer should \emph{minimize information access cost} by providing immediate visual feedback and minimizing the effort required to create legible visualizations; and (3) based on the \emph{principle of pictorial realism} a visualizer should provide functionality for creating domain specific visualizations.

Consistent with design goals 1 and 2, Sterling is a web application, packaged with a custom build of Alloy, that is launched automatically in the user's default browser by Alloy when an instance is generated. 
% Drawing on the approach taken by tools like Alloy4Fun~\cite{macedo2019} and BMotionWeb~\cite{ladenberger2016}, a web-based platform was chosen due to the availability of robust data visualization and user interface libraries as well as the popularity of the JavaScript programming language.
% A client-server relationship is established between Sterling and Alloy by an embedded web server in Alloy, enabling instances to be delivered to Sterling in the XML format provided by Alloy. 
The user interface is similar to Alloy's own, providing graph and table views which extend the functionality of their counterparts in Alloy, while the addition of a ``script'' view provides users with the ability to create custom visualizations from instance data by writing JavaScript code, addressing our third design goal.
Communication between Alloy and the individual views is managed using a mediator pattern, illustrated in Figure~\ref{fig:communication}.

\begin{figure}
    \centering
    \dummyfigsmall{Communication pattern image.}
    \caption{The Sterling communication pattern.}
    \label{fig:communication}
\end{figure}


The Sterling graph view provides all of the same functionality of the Alloy graph view, but is supported by a few key extensions. Most notably, graph elements are not restricted to rows; users may freely arrange graphical elements to make the display more readable. Furthermore, the layout algorithm is not automatically executed when the projection is updated or when a new instance is generated, unless the graphs do not have any elements in common. As such, graphical elements remain static as users step through stateful models and generate instances.

The Sterling script view provides an environment for the rapid development of custom visualizations by bringing together three components: a text editor, a blank rendering stage, and a JavaScript execution environment. 
This combination gives the user a basic ``code sandbox'' in which they can use their favorite JavaScript libraries to create visualizations based on the instance data. Within the script view environment, all instance data---the signatures, fields, atoms, and tuples---are exposed as JavaScript variables. 
Furthermore, users have direct access to the npm package repository~\cite{npm} which can be used to add visualization (or any other useful) libraries to the scripting environment.
This combination enables, for example, a user to bind atoms to SVG circles using the D3 visualization library, and to calculate their positions based on the relationships defined by the tuples.
In our own use we have found this paradigm to be particularly useful for visualizing instances of models that have inherent spatial properties, such as those shown in Figure~\ref{fig:script}.
Scripts written in the editor are executed each time Sterling receives a new instance or the projection is updated. 
As such, users enjoy the same type of immediate visual feedback provided by the graph view with the added benefit of complete control over the visual abstraction used to present the instance.

\begin{figure}
    \centering
    \dummyfiglarge{Script View Example}
    \caption{Custom visualizations created with the Sterling script view: (1) an instance of the dining philosophers problem in which deadlock has been reached; (2) an instance of a binary tree model in which nodes are positioned based on their presence in the ``left'' and ``right'' relations.}
    \label{fig:script}
\end{figure}

%Sterling is a web application, packaged with a custom build of Alloy, that provides both an enhanced graph visualization to address limitations of Alloy's visualizer and a means for creating custom visualizations when the directed graph representation proves insufficient.
%Drawing on the approach taken by tools like Alloy4Fun~\cite{macedo2019} and BMotionWeb~\cite{ladenberger2016}, a web-based platform was chosen due to the availability of robust data visualization and user interface libraries as well as the popularity of the JavaScript programming language. 
%Consistent with design goals~\ref{dg1} and~\ref{dg2}, Sterling is launched automatically by Alloy when an instance is generated, presenting a familiar user interface. Mirroring the Alloy Visualizer interface, multiple methods for viewing the instance are available through 

\section{Conclusions and Ongoing Work}
\label{conclusions}

Include a link to the demo!

Furthermore, the mediator pattern provides flexibility in terms of data providers and visualizations. Regarding data providers, Alloy need not be the source of data so long as the data being supplied to Sterling adheres to the XML format of an Alloy instance and communication is in adherence with the protocol outlined above. As such, any model finder can be used to supply data for visualization. Indeed, an Alloy-like model finder called Forge, developed at Brown University fora Logic for Systems class, makes use of Sterling for visualizations.

\bibliographystyle{splncs04}
\bibliography{main}

\end{document}
\documentclass[runningheads]{llncs}

\newcommand{\dummyfigsmall}[1]{
  \centering
  \fbox{
    \begin{minipage}[c][0.15\textheight][c]{0.5\textwidth}
      \centering{#1}
    \end{minipage}
  }
}

\newcommand{\dummyfiglarge}[1]{
  \centering
  \fbox{
    \begin{minipage}[c][0.35\textheight][c]{0.5\textwidth}
      \centering{#1}
    \end{minipage}
  }
}

\begin{document}

\title{Sterling: A Web-based Visualizer for Alloy}
\author{Tristan Dyer \inst{1} \and John Baugh \inst{2}}
\authorrunning{Dyer and Baugh}

\institute{Brown University \and North Carolina State University}

\authorrunning{Dyer and Baugh}

\maketitle

\begin{abstract}
We introduce Sterling, a web-based visualization tool for Alloy that provides enhanced versions of existing Alloy visualizations and introduces new capabilities to address shortcomings of the visual abstractions typically used to display instances of relational models. 
%While development is driven primarily by the needs of the Alloy community, other relational logic and model finding tools can also be accommodated in its model-view architecture, which uses a mediator pattern to keep the approach data agnostic.
\end{abstract}

\section{Introduction}
\label{introduction}

Model finding tools like Alloy~\cite{jackson2012} enable a lightweight approach to design and reasoning about complex software systems. Such tools provide push-button analysis for both checking assertions within bounded scopes, and for generating instances that satisfy a property of interest. An attractive feature of Alloy is the immediate feedback provided by visualizations, allowing users to inspect instances and counterexamples in order to identify design problems. The ability to communicate visual information \emph{intuitively} therefore plays a key role in determining the effectiveness of interactions with the user~\cite{gammaitoni2014}.

\section{Visualization in Alloy}
\label{alloy-vis}

The Alloy Analyzer includes a visualizer that can display an instance in one of four possible views: text, table, tree, and, perhaps the most frequently used, graph. In that view, the instance is rendered as a directed graph in which nodes represent atoms and edges represent tuples of relations. Basic properties of the graph, such as labels, visibility, color, and shape, can be customized manually or by using themes.
This customizability, in combination with the ability to manually reposition nodes, allows the user to interactively explore the graph so that they can better understand the instance. 
Additionally, the graph view supports ``projection'', a feature most commonly used with models of dynamic systems. When an instance of such a model is projected over time, the user is able to interactively step through snapshots of individual states in sequence.

Despite these features, certain limitations of the visualizer make it difficult to use as instances increase in size and complexity. 
Issues identified in the literature include the layout algorithm, which provides no option for customization by the user and rigidly arranges graph nodes into rows~\cite{couto2018,macedo2019}, and the low-level nature of the graph representation, which can become difficult to interpret as models become more complex~\cite{gammaitoni2014}. 
Furthermore, the graph layout is recalculated any time a new instance is generated or the projection is changed, and so the user must reinterpret the entire graph if, for example, they are stepping through the state atoms in sequence~\cite{couto2018,misue1995,zaman2013}.

%Furthermore, our own experience using Alloy in the field of scientific computing has highlighted the need for a visualization approach capable of expressing \emph{spatial} relationships---not just topological ones---and to \emph{consistency} in those relationships when dynamic updates occur, as they do in problems with time-varying state.

Various approaches have been proposed to address these and other issues, either by extending Alloy's visualizer or by introducing new tools. Those extending the existing visualizer may, for instance, improve the visualization by automatically determining projections and attributes using contextual clues from the instance~\cite{zaman2013}. Those introducing new tools may, e.g., provide a wider range of layout algorithms~\cite{couto2018,macedo2019}, or extend the Alloy language so the user can provide visualization directives as part of the model~\cite{gammaitoni2014}.

\section{Sterling Design Goals and Architecture}
\label{sterling}

The following design goals define our approach to instance visualization based on both the strengths and shortcomings of the existing visualizations, and draw inspiration from the Thirteen Principles of Display Design~\cite{}, as indicated by \emph{italics}.

\begin{enumerate}
    \item \label{dg1} Based on the \emph{principle of consistency}, a tool supporting Alloy visualization by extending existing functionality should provide a user interface similar to that of the Alloy visualizer when possible.
    \item \label{dg2} A tool should \emph{minimize information access cost} by providing immediate visual feedback and minimizing the effort required to create legible visualizations.
    \item \label{dg3} Based on the \emph{principle of pictorial realism}, which states that a display should look like the variable that it represents, a tool should provide functionality for creating domain specific visualizations.
\end{enumerate}

Consistent with design goals~\ref{dg1} and~\ref{dg2}, Sterling is a web application, packaged with a custom build of Alloy, that provides a user interface similar to Alloy's own and is launched automatically by Alloy when an instance is generated. 
Drawing on the approach taken by tools like Alloy4Fun~\cite{macedo2019} and BMotionWeb~\cite{ladenberger2016}, a web-based platform was chosen due to the availability of robust data visualization and user interface libraries as well as the popularity of the JavaScript programming language.
A client-server relationship is established between Sterling and Alloy by an embedded web server in Alloy, enabling instances to be delivered to Sterling in the XML format provided by Alloy. 
As in the Alloy visualizer, the Sterling user interface is organized into separate ``views'', each providing a different method for visualizing the instance. 
Communication between Alloy and the individual views is managed using a mediator pattern, illustrated in Figure~\ref{fig:communication}. Three views are packaged with Sterling: graph, table, and script. The first two views extend the functionality of their counterparts in Alloy, while the script view, addressing design goal~\ref{dg3}, provides users with the ability to create custom visualizations from instance data by writing JavaScript code.

\begin{figure}
    \centering
    \dummyfigsmall{Communication pattern image.}
    \caption{The Sterling communication pattern.}
    \label{fig:communication}
\end{figure}


The Sterling graph view provides all of the same functionality of the Alloy graph view, but is supported by a few key extensions. Most notably, graph elements are not restricted to rows; users may freely arrange graphical elements to make the display more readable. Furthermore, the layout algorithm is not automatically executed when the projection is updated or when a new instance is generated, unless the graphs do not have any elements in common. As such, graphical elements remain static as users step through stateful models and generate instances.

Paragraph about the script view and how it works.

\begin{figure}
    \centering
    \dummyfiglarge{Script View Example}
    \caption{Script view example with graph view on left, custom vis on right.}
    \label{fig:script}
\end{figure}

%Sterling is a web application, packaged with a custom build of Alloy, that provides both an enhanced graph visualization to address limitations of Alloy's visualizer and a means for creating custom visualizations when the directed graph representation proves insufficient.
%Drawing on the approach taken by tools like Alloy4Fun~\cite{macedo2019} and BMotionWeb~\cite{ladenberger2016}, a web-based platform was chosen due to the availability of robust data visualization and user interface libraries as well as the popularity of the JavaScript programming language. 
%Consistent with design goals~\ref{dg1} and~\ref{dg2}, Sterling is launched automatically by Alloy when an instance is generated, presenting a familiar user interface. Mirroring the Alloy Visualizer interface, multiple methods for viewing the instance are available through 

\section{Conclusions and Ongoing Work}
\label{conclusions}

Furthermore, the mediator pattern provides flexibility in terms of data providers and visualizations. Regarding data providers, Alloy need not be the source of data so long as the data being supplied to Sterling adheres to the XML format of an Alloy instance and communication is in adherence with the protocol outlined above. As such, any model finder can be used to supply data for visualization. Indeed, an Alloy-like model finder called Forge, developed at Brown University fora Logic for Systems class, makes use of Sterling for visualizations.

\bibliographystyle{splncs04}
\bibliography{main}

\end{document}